%%%%%%%%%%%%%%%%%%%%%%%%%%%%%%%%%%%%%%%%%
% Awesome Resume/CV
% XeLaTeX Template
% Version 1.1 (9/1/2016)
%
% This template has been downloaded from:
% http://www.LaTeXTemplates.com
%
% Original author:
% Claud D. Park (posquit0.bj@gmail.com) with modifications by
% Vel (vel@latextemplates.com)
%
% License:
% CC BY-NC-SA 3.0 (http://creativecommons.org/licenses/by-nc-sa/3.0/)
%
% Important note:
% This template must be compiled with XeLaTeX, the below lines will ensure this
%!TEX TS-program = xelatex
%!TEX encoding = UTF-8 Unicode
%
%%%%%%%%%%%%%%%%%%%%%%%%%%%%%%%%%%%%%%%%%
%----------------------------------------------------------------------------------------
%   PACKAGES AND OTHER DOCUMENT CONFIGURATIONS
%----------------------------------------------------------------------------------------
\documentclass[12pt, a4paper]{awesome-cv}

\geometry{left=2cm, top=1.5cm, right=2cm, bottom=2cm, footskip=.5cm} % Configure page margins with geometry

%%% Location of fonts
\fontdir[fonts/]

%%% Configure a directory location for sections
% \newcommand*{\sectiondir}{resume/}

%%% Override color
% Awesome Colors: awesome-emerald, awesome-skyblue, awesome-red, awesome-pink, awesome-orange
%                 awesome-nephritis, awesome-concrete, awesome-darknight
%% Color for highlight
% Define your custom color if you don't like awesome colors
\colorlet{awesome}{awesome-concrete}
%\definecolor{awesome}{HTML}{CA63A8}
%% Colors for text
%\definecolor{darktext}{HTML}{414141}
%\definecolor{text}{HTML}{414141}
%\definecolor{graytext}{HTML}{414141}
%\definecolor{lighttext}{HTML}{414141}

%%% Override a separator for social informations in header(default: ' | ')
%\headersocialsep[\quad\textbar\quad]
%%%%%%%%%%%%%%%%%%%%%%%%%%%%%%%%%%%%%%
%     3rd party packages
%%%%%%%%%%%%%%%%%%%%%%%%%%%%%%%%%%%%%%
%%% Needed to divide into several files
\usepackage{import}
\usepackage{hyperref}

%%%%%%%%%%%%%%%%%%%%%%%%%%%%%%%%%%%%%%
%     Personal Data
%%%%%%%%%%%%%%%%%%%%%%%%%%%%%%%%%%%%%%
%%% Essentials
\name{Prakhar}{Agarwal}
% \address{}
\mobile{(+91)9468652335} 
%%% Social
\email{pkhrag@iitk.ac.in}
\homepage{home.iitk.ac.in/~pkhrag}
\github{pkhrag}
%\linkedin{posquit0}
%%% Optionals
\position{THIRD YEAR UNDERGRADUATE{\enskip\cdotp\enskip}Computer Science and Engineering, IIT Kanpur}
%%%%%%%%%%%%%%%%%%%%%%%%%%%%%%%%%%%%%%
%     Content
%%%%%%%%%%%%%%%%%%%%%%%%%%%%%%%%%%%%%%
%%% Make a footer for CV with three arguments(<left>, <center>, <right>)
\makecvfooter
  {\today}
  {Prakhar Agarwal~~~·~~~Résumé}
  {\thepage}
\begin{document}
%%% Make a header for CV with personal data
\makecvheader
% %%% Import contents
% \import{\sectiondir}{education.tex}
%%%%%%%%%%%%%%%%%%%%%%%%%%%%%%%%%%%%%%%%%%%%%%%%%%%%%%%%%%%%%%%%%%%%%%%
\cvsection{Education}
% {\fontsize{11pt}{1em}\bodyfontlight\upshape\color{text}
\begin{cventries}
  \cventry
    {Bachelor of Technology, Computer Science and Engineering}
    {IITK (Indian Institute of Technology, Kanpur)}
    {Kanpur, India}
    {2015 - 2019 (Expected)}
    {
      \begin{cvitems}
        \item {Cumulative Grade Point/ CPI:\textbf{ 9.64/10.0} }
      \end{cvitems}
    }
    \vspace{0.2cm}
  \cventry
    {RBSE}
    {Central Public School}
    {Rajasthan, India}
    {2013}
    {
      \begin{cvitems}
        \item {$12 ^{th}$ GRADE | Aggregate \textbf{89.8\%}}      
      \end{cvitems} 
    }
    \vspace{0.2cm}
  \cventry
    {CBSE}
    {Kendriya Vidyalaya, Jhalawar}
    {Rajasthan, India}
    {2013}
    {
      \begin{cvitems}
        \item {$10 ^{th}$ GRADE | CGPA \textbf{10.0/10.0}}      
      \end{cvitems} 
    }

\end{cventries}
%%%%%%%%%%%%%%%%%%%%%%%%%%%%%%%%%%%%%%%%%%%%%%%%%%%%%%%%%%%%%%%%%%%%%%%%%%%%%%%%%%%%
\vspace{-0.3cm}
\cvsection{Honors \& Awards}
% {\fontsize{11pt}{1em}\bodyfontlight\upshape\color{text}
\begin{cvhonors}
  \cvhonor
    {2nd Rank}
    {Championship Value Prediction, ISCA}
    {Los Angeles}
    {2018}
  \cvhonor
    {All India Rank 126}
    {among 1.2 million students in JEE ADVANCED}
    {India}
    {2015}
  \cvhonor
    {Academic Excellence Award}
    {for two consecutive years in IIT Kanpur}
    {Kanpur, India}
    {2015-17}
  \cvhonor
    {0.1 Percentile}
    {JEE MAINS}
    {India}
    {2015}
  \cvhonor
    {KVPY Fellow | All India Rank 232}
    {IISc Bangalore and Government of India}
    {Bangalore, India}
    {2014}
  \cvhonor
    {1 Percentile}
    {Physics Olympiad (NSEP) | HBCSE}    
    {Rajasthan, India}
    {2014}  
  \cvhonor
    {Qualified}
    {Chemistry Olympiad (NSEC) | HBCSE}
    {Rajasthan, India}
    {2014}

\end{cvhonors}
%%%%%%%%%%%%%%%%%%%%%%%%%%%%%%%%%%%%%%%%%%%%%%%%%%%%%%%%%%%%%%%%%%%%%%%

\cvsection{Publication}
\begin{cventries}
    \cventry
    {International Symposium on Computer Architecture}
    {DFCM++: Augmenting DFCM with Early Update and Data Dependency-driven Value Estimation}
    {Los Angeles, California, US}
    {June 2018}
    {Placed 2nd in Championship Value Prediction Workshop}
\end{cventries}

\cvsection{Work Experience}
\begin{cventries}
    \cventry
    {Cloud Lab}
    {Samsung Electronics Headquarters}
    {Suwon, South Korea}
    {May 2018 - July 2018}
    {
        \begin{cvitems}
        \item Worked on prototype for Event Monitoring and Corrective Action for network functions of Samsung.
        \item Worked on a simulator of the Data Collection, Analytics and Events (DCAE) module of Open Network Automation Platform(ONAP).
        \item Used various APIs like REST-API, Kafka-python, Durable-rules, etc. to develop the simulator.
        \end{cvitems}
    }

    \cventry
    {IT Department, Security Team}
    {Bombay Stock Exchange}
    {Bombay, India}
    {May 2017 - July 2017}
    {
        \begin{cvitems}
            \item Worked on designing and improving Network security architecture of Bombay Stock Exchange with IBM. 
            \item Found vulnerabilities and audited BSE network with SEBI compliance. 
        \end{cvitems}
    }

    \cventry
    {Tutor, Introduction to Computing (ESC101)}
    {IIT Kanpur}
    {Kanpur, India}
    {July 2018 - Present}
    {
        \begin{cvitems}
            \item An introductory course in C and programming techniques, with more than 450 enrolled students.
            \item Conducting Tutorial Lectures for a batch of 40 students, once every week.
            \item Designing and Grading Lab Sessions, Quizzes, Theory and Lab exams.
            \item Supervising the work of Teaching Assistants.
        \end{cvitems}
    }
\end{cventries}

\cvsection{Projects}
% {\fontsize{11pt}{1em}\bodyfontlight\upshape\color{text}

\begin{cventries}
  \cventry
    {Project, Prof. Biswabandan Panda}
    {Side-Channel Attacks exploiting Speculative Execution}
    {IIT Kanpur}
    {December 2017 - Present}
    {
      \begin{cvitems}
      \item The aim of the project is to discover new side-channel attacks using various speculative techniques like cache, prefetching, out-of-order execution, value prediction, branch prediction, etc.
      \item Mounted Flush-Reload, Flush-Flush, and other  side-channel attacks on current processors.
      \item Using gem5 simulator to explore new side-channel attacks based on value and branch prediction.
      \end{cvitems}
    }
    \vspace{0.2cm}

  \cventry
    {Member, Team AUV-IITK, Prof. K.S. Venkatesh \& Prof. Sachin Shinde}
    {Autonomous Underwater Vehicle\ \ \ \texttt{\href{https://github.com/AUV-IITK}{github.com/AUV-IITK}}\ \ \ \texttt{\href{https://drive.google.com/file/d/0B8_5kUnCr-WzbnRGV01lcEVTWDA/view?usp=sharing}{REPORT}} }
    {IIT Kanpur}
    {December 2015 - April 2017}
    {
      \begin{cvitems}
        \item {Leader Software Subsystem of team AUV-IITK.}
        \item {First Runner-ups at the National level Competition SAVe-NIOT (2016) in debut attempt.}
        \item {The aim of the Robot is detecting and following a line, hitting a buoy, firing torpedoes at given targets and dropping objects underwater.}
        \item {Implemented the tasks using Robot Operating System (ROS) for Controls, OpenCV for Image Processing and Gazebo for simulation.}        
      \end{cvitems}
    }

  \cventry
    {Course Project, Prof. Sandeep Shukla}
    {Securing Zoobar Server}
    {IIT Kanpur}
    {Jan 2017 - April 2017}
    {
      \begin{cvitems}
      \item Found various security vulnerabilities in the server code.
      \item Exploited vulnerabilities using control hijacking techniques like buffer overflow and format string attacks.
      \item Performed various browser-based attacks like SQL injection, XSS, CSRF on Zoobar web application.
      \item Fixed security bugs in web server, implemented privilege separation and server-side sand-boxing of executable profiles in the okws server.
      \end{cvitems}
    }

  \cventry
  {Course Project, Prof. Purushottam Kar}
  {Text Generation}
  {IIT Kanpur}
  {August 2017 - December 2017}
  {
      \begin{cvitems}
              \item The aim of our project is to capture the artistic style of an author and then generate text in that same style.
              \item We use character-level language models using multi-layered RNNs/LSTM and attention models on top of that to account for long-term dependencies.
              \item As an alternative approach, we also aim to use neural style transfer for text in a fashion similar to artistic style transfer for images by Gatys et al.
      \end{cvitems}
  }

  \cventry
    {Semester Project, ACA}
    {Introduction to Game Theory\ \ \   \texttt{\href{https://github.com/pkhrag/ACA-Project-Game-Theory}{github.com/pkhrag/Game-Theory}}}
    {IIT Kanpur}	
    {January 2016 - April 2016}
    {
      \begin{cvitems}
    	\item Developed an AI for zero-sum games like Tic-Tac-Toe and Connect-4.
    	\item Used min-max algorithm with alpha-beta optimization and heuristic functions.
        \item Explored Monte Carlo Simulation for computing better heuristic function.
    	\item Implemented feature of connecting two AIs over a network using ZeroMQ library.
      \end{cvitems}
    }

\end{cventries}
\cvsection{Other Projects}
\vspace{-0.3cm}
{\fontsize{12pt}{1em}\bodyfontlight\upshape\color{text}
\begin{tabular}{l l}
 Compiler for Golang to x86& Exploiting Media Projection Vulnerability in Android\\
    Implementing OS Functionality in NachOS & Centralized File Storage System\\ 
 
 \end{tabular}
 }
%%%%%%%%%%%%%%%%%%%%%%%%%%%%%%%%%%%%%%%%%%%%%%%%%%%%%%%%%%%%%%%%%%%%%%%%%%%%%%%%%
\cvsection{Relevant Courses}
\vspace{-0.3cm}
{\fontsize{11pt}{1em}\bodyfontlight\upshape\color{text}
\begin{tabular}{l l l}
\textbf{\textit{Completed}}\\
 Computer System Security & Computer Organization & Linear Algebra\\ 
 Operating Systems* & Advanced Algorithms* & Computational Complexity Theory\\
 Machine Learning Techniques & Computer Networks & Theory of Computation\\
 Data Structures and Algorithms & Probability and Statistics* & Partial Differential Equation\\
 Discrete Mathematics  & Multivariable Calculus & Abstract Algebra\\
 Introduction to Electronics & Logic in Computer Science & Introduction to Programming \\
 Computer Laboratory & Introduction to Economics & Computer Architecture*\\
 Compiler Design\\
 
\textbf{\textit{Ongoing}}\\
 Secure Memory Systems & Cyber Security of Critical Infrastructures\\

 {\footnotesize  * : A* Grade for exceptional performance}\\
 \end{tabular}
 }
% {\fontsize{11pt}{1em}\footerfont\upshape\color{text}
% \begin{tabular*}{\textwidth}{L{8.4cm} L{3cm} L{3.5cm}}
%   \entrylocationstyle{$*$: A* Grade for exceptional performance(Awarded to top $1\%$ of the class)}\\
% \end{tabular*}
% }
\vspace{-0.3cm}
% \begin{tabular}{l l l}
% \textbf{\textit{Completed}}\\
%  Introduction to Programming* & Partial Differential Equation & Linear Algebra*\\ 
%  Introduction to Electrodynamics & Multivariable Calculus* & Abstract Algebra\\
%  Introduction to Classical Mechanics & Discrete Mathematics & Thermodynamics\\
%  Introduction to Electronics & Logic in Computer Science\\
% \textbf{\textit{Ongoing}}\\
%  Data Structures and Algorithms & Probability and Statistics & Computer Organization\\
%  System Security & Machine Learning(Coursera)\\ 
%  \vspace{0.4cm}
%  {\footnotesize * : Grade for Exceptional Performance }
%  \end{tabular}
%  %%%%%%%%%%%%%%%%%%%%%%%%%%%%%%%%%%%%%%%%%%%%%%%%%%%%%%%%%%%%%%%%%%%%%%%%%%%%%%%%
%%%%%%%%%%%%%%%%%%%%%%%%%%%%%%%%%%%%%%%%%%%%%%%%%%%%%%%%%%%%%%%%%%%%%%%%%%%%%%%%%

\cvsection{Skills}
% {\fontsize{13pt}{1em}\bodyfontlight\upshape\color{text}
\begin{cvskills}

  % ------------------------------------------------

  \cvskill
  {Programming}
  {C/C++, Python, X86 Assembly, ROS, Octave, MIPS, Verilog, R, Golang, Java }

  % ------------------------------------------------

  \cvskill
  {Web}
  {JavaScript, HTML, CSS, SQL}

  % ------------------------------------------------

  \cvskill
  {Utilities}
  {Bash, Git, GDB, \LaTeX, Sublime, Vim, Numpy, OpenCV}
  
  % ------------------------------------------------
  \cvskill
  {Platforms}
  {Windows, Linux(Ubuntu), Arduino, Android, FPGA, Gem5}
\end{cvskills}



\cvsection{Positions of Responsibility}
% {\fontsize{11pt}{1em}\bodyfontlight\upshape\color{text}
\begin{cventries}

  \cventry
  {Departmental Student Body}
  {ACA Coordinator}
  {IITK}
  {January 2017 - Present}
  {
    \begin{cvitems}
    \item Conducted various events in the department like CTFs, Hackathons, and other competitions.
    \item Organized departmental trip, departmental freshers, farewell and happy hour.
    \end{cvitems}
  }
    \vspace{0.2cm}
  \cventry
    {Academic Mentor}
    {Counselling Service (IITK)}
    {IITK}
    {August 2016 - July 2017}
    {
      \begin{cvitems}
        \item {Helped students to cope up with academics.}
        \item {Took institute level doubt sessions for Fundamentals of Computing course.}
      \end{cvitems}
    }
    \vspace{0.2cm}
  \cventry
  {Project Mentor}
  {Programming Club, IITK}
  {IITK}
  {May 2017 - July 2017}
  {
    \begin{cvitems}
    \item Mentored 20 students in a project Ethical Hacking and Cyber Security.
    \item Designed a CTF for intra-IITK Science and Technology Championship.
    \end{cvitems}
  }
\end{cventries}
%%%%%%%%%%%%%%%%%%%%%%%%%%%%%%%%%%%%%%%%%%%%%%%%%%%%%%%%%%%%%%%%%%%%%%%%%%%%%%%%%%
% \cvsection{Miscellaneous}
% % {\fontsize{11pt}{eh1em}\bodyfontlight\upshape\color{text}
% \begin{cvhonors}
%   \cvhonor
%     {Presentation}
%     {on Diophantine Equations and Chinese Remainder Theorem}
%     {IITK}
%     {2016}

% \end{cvhonors}
% %%%%%%%%%%%%%%%%%%%%%%%%%%%%%%%%%%%%%%%%%%%%%%%%%%%%%%%%%%%%%%%%%%%%%%%%%%%%%%%%%
\end{document}
