% !TEX TS-program = xelatex
% !TEX encoding = UTF-8

%%%%%%%%%%%%%%%%%%%%%%%%%%%%%%%%%%%%%%%%%%%%%%%%%%%%%%%%%%%%%%%%%
%% SIMPLE-RESUME-CV
%% <https://github.com/zachscrivena/simple-resume-cv>
%% This is free and unencumbered software released into the
%% public domain; see <http://unlicense.org> for details.
%%%%%%%%%%%%%%%%%%%%%%%%%%%%%%%%%%%%%%%%%%%%%%%%%%%%%%%%%%%%%%%%%

%%%%%%%%%%%%%%%%%%%%%%%%%%%%%%%%%%%%%%%%%%%%%%%%%%%%%%%%%%%%%%%%%
%% INSTRUCTIONS FOR COMPILING THIS DOCUMENT ("CV.tex")
%% TeX ---(XeLaTeX)---> PDF:
%%
%% Method 1: Use latexmk for fully automated document generation:
%%   latexmk -xelatex "CV.tex"
%%   (add the -pvc switch to automatically recompile on changes)
%%
%% Method 2: Use XeLaTeX directly:
%%   xelatex "CV.tex"
%%   (run multiple times to resolve cross-references if needed)
%%%%%%%%%%%%%%%%%%%%%%%%%%%%%%%%%%%%%%%%%%%%%%%%%%%%%%%%%%%%%%%%%

% \documentclass[a4paper,10pt,oneside]{article}
\documentclass[letterpaper,10pt,oneside]{article}
\usepackage[super]{nth}
%%%%%%%%%%%%%%%%%%%%%%%%%%%%%%%%%%%%%%%%%%%%%%%%%%%%%%%%%%%%%%%%%
%% TYPESETTING OPTIONS.
%%%%%%%%%%%%%%%%%%%%%%%%%%%%%%%%%%%%%%%%%%%%%%%%%%%%%%%%%%%%%%%%%

\newcommand{\TypesetInNonStopMode}{1}
\newcommand{\TypesetInDraftMode}{0}

%%%%%%%%%%%%%%%%%%%%%%%%%%%%%%%%%%%%%%%%%%%%%%%%%%%%%%%%%%%%%%%%%
%% PREAMBLE.
%%%%%%%%%%%%%%%%%%%%%%%%%%%%%%%%%%%%%%%%%%%%%%%%%%%%%%%%%%%%%%%%%

%%%%%%%%%%%%%%%%%%%%%%%%%%%%%%%%%%%%%%%%%%%%%%%%%%%%%%%%%%%%%%%%%
%% SIMPLE-RESUME-CV
%% <https://github.com/zachscrivena/simple-resume-cv>
%% This is free and unencumbered software released into the
%% public domain; see <http://unlicense.org> for details.
%%%%%%%%%%%%%%%%%%%%%%%%%%%%%%%%%%%%%%%%%%%%%%%%%%%%%%%%%%%%%%%%%

% Run in non-stop mode.
\ifnum\TypesetInNonStopMode=1
\nonstopmode
\fi

% Geometry package for page margins.
\usepackage[
left=0.70in,
right=0.70in,
top=0.60in,
bottom=0.45in,
nohead,
includefoot]{geometry}

% PDF settings and properties.
\usepackage{hyperref}

% Long table for page layout.
\usepackage{longtable}

% Hyphenation: Disabled.
\usepackage[none]{hyphenat}

% Colors.
\usepackage[usenames]{color}
% \definecolor{MyDarkBlue}{RGB}{0,90,160}
% {\color{MyDarkBlue}This text is dark blue}

% Current date and time.
\usepackage[yyyymmdd,24hr]{datetime}
\renewcommand{\dateseparator}{-}
\settimeformat{xxivtime}
% {\today}~{\currenttime}

% Timestamp.
\newcommand{\Timestamp}{{\yyyymmdddate\today}~{\currenttime}}

% Abbreviations for months.
\newcommand{\LongMonth}[1]{%
\ifcase#1\relax
\or January%
\or February%
\or March%
\or April%
\or May%
\or June%
\or July%
\or August%
\or September%
\or October%
\or November%
\or December%
\fi}
\newcommand{\ShortMonth}[1]{%
\ifcase#1\relax
\or Jan%
\or Feb%
\or Mar%
\or Apr%
\or May%
\or Jun%
\or Jul%
\or Aug%
\or Sep%
\or Oct%
\or Nov%
\or Dec%
\fi}

% Select datestamp format.
\def\DatestampFormatSelection{2}

% Datestamp format: {yyyy}{MM}{dd} ---> yyyy-MM-dd (e.g., 2010-12-31).
\ifnum\DatestampFormatSelection=1
\newcommand{\DatestampYMD}[3]{\mbox{#1-#2-#3}}
\newcommand{\DatestampYM}[2]{\mbox{#1-#2}}
\newcommand{\DatestampY}[1]{#1}
\fi

% Datestamp format: {yyyy}{MM}{dd} ---> MMM yyyy (e.g., Dec 2010).
\ifnum\DatestampFormatSelection=2
\newcommand{\DatestampYMD}[3]{\mbox{\ShortMonth{#2} #1}}
\newcommand{\DatestampYM}[2]{\mbox{\ShortMonth{#2} #1}}
\newcommand{\DatestampY}[1]{#1}
\fi

% Datestamp format: {yyyy}{MM}{dd} ---> MMMM yyyy (e.g., December 2010).
\ifnum\DatestampFormatSelection=3
\newcommand{\DatestampYMD}[3]{\mbox{\LongMonth{#2} #1}}
\newcommand{\DatestampYM}[2]{\mbox{\LongMonth{#2} #1}}
\newcommand{\DatestampY}[1]{#1}
\fi

% Datestamp format: {yyyy}{MM}{dd} ---> yyyy (e.g., 2010).
\ifnum\DatestampFormatSelection=4
\newcommand{\DatestampYMD}[3]{#1}
\newcommand{\DatestampYM}[2]{#1}
\newcommand{\DatestampY}[1]{#1}
\fi

% XeLaTeX packages.
\usepackage{fontspec}
\defaultfontfeatures{Ligatures=TeX}
\usepackage{xunicode}
\usepackage{xltxtra}

% Font: Use "Tinos" as the main typeface (\textnormal{}, \normalfont).
% The "Tinos" fonts are released under the Apache License Version 2.0,
% and can be downloaded for free at <http://www.fontsquirrel.com/fonts/tinos>.
% Symbol table: <http://www.fileformat.info/info/unicode/font/tinos/grid.htm>
\setmainfont
[Path=./Fonts/Tinos/,
ItalicFont=Tinos-Italic,
BoldFont=Tinos-Bold,
BoldItalicFont=Tinos-BoldItalic]
{Tinos-Regular.ttf}

% Secondary font: "GNU FreeFont".
% The "GNU FreeFont" fonts are released under the
% GNU General Public License Version 3, and can be downloaded
% for free at <https://savannah.gnu.org/projects/freefont/>.
\newcommand{\UseSecondaryFont}{\fontspec
[Path=./Fonts/GNUFreeFont/,
ItalicFont=FreeSerifItalic,
BoldFont=FreeSerifBold,
BoldItalicFont=FreeSerifBoldItalic]
{FreeSerif.otf}}

% Sans-serif font: Changed to "Tinos".
\renewcommand{\sffamily}{\rmfamily}

% Typewriter (monospace) font: Changed to "Tinos".
\renewcommand{\ttfamily}{\rmfamily}

% Small caps font: Changed to "Tinos".
\renewcommand{\scshape}{\rmfamily}
%\renewcommand{\textsc}[1]{\textbf{\MakeUppercase{#1}}}

% Font styles.
\newcommand{\UseHeadingFont}{\normalfont}
\newcommand{\UseHeaderFooterFont}{\UseHeadingFont\fontsize{8.2pt}{9.5pt}\selectfont}
\newcommand{\UseNoteFont}{\UseHeadingFont\fontsize{8pt}{9.6pt}\selectfont}
\newcommand{\UseTitleFont}{\UseHeadingFont\fontsize{28pt}{33.6pt}\selectfont\bfseries}
\newcommand{\UseSubTitleFont}{\normalfont\fontsize{8.6pt}{10.3pt}\selectfont}
\newcommand{\UseSectionFont}{\UseHeadingFont\fontsize{9pt}{11pt}\selectfont\bfseries}
\newcommand{\UseSubSectionFont}{\UseHeadingFont\fontsize{8.6pt}{10.3pt}\selectfont\bfseries}
\newcommand{\UseDetailFont}{\normalfont\fontsize{8.6pt}{10.3pt}\selectfont}

% Symbols (unicode).
\newcommand{\BulletSymbol}{{\normalfont\fontsize{6.5}{7.8}\selectfont\raisebox{0.17em}{\char"25A0}}}
\newcommand{\SubBulletSymbol}{{\normalfont\fontsize{6}{7.2}\selectfont\raisebox{0.17em}{\char"25CF}}}
\newcommand{\TildeSymbol}{{\normalfont\char"007E}}

% Headers and footers: Blank header, page number in footer.
\ifnum\TypesetInDraftMode=0
\newcommand{\HeaderText}{}
\newcommand{\FooterText}{\UseHeaderFooterFont\hfill%
{Page}~{\thepage}~of~\pageref{LastPage}%
\hfill}
\else
\newcommand{\HeaderText}{}
\newcommand{\FooterText}{\UseHeaderFooterFont%
\hphantom{DRAFT~\Timestamp}\hfill%
{Page}~{\thepage}~of~\pageref{LastPage}%
\hfill{\color{red}DRAFT~\Timestamp}}
\fi

\makeatletter
\def\ps@plain{%
\def\@oddhead{\HeaderText}%
\def\@evenhead{\HeaderText}%
\def\@oddfoot{\FooterText}%
\def\@evenfoot{\FooterText}}
\makeatother

\pagestyle{plain}

% Paragraph style.
\setlength{\parindent}{0in} % No indentation at the beginning of each paragraph.
\setlength{\parskip}{0in} % No vertical space between paragraphs.

% Footnotes: Use symbols instead of numbers for labels.
\renewcommand{\thefootnote}{\fnsymbol{footnote}}

% Macro: title (name).
\renewcommand{\title}[1]{%
\pdfbookmark[1]{#1}{#1}%
\par\begin{center}%
\par\UseTitleFont%
{#1}%
\par\end{center}%
\par\vspace{-1.75em}\par}

% Macro: subtitle (personal information below name).
\newenvironment{subtitle}
{\par\begin{center}%
\par\UseSubTitleFont}
{\par\end{center}\par}

% Macro: body (rest of the document).
\newenvironment{body}
{\par\vspace{-1em}\par
\begin{longtable}{p{0.15\textwidth}p{0.80\textwidth}}}
{\par\end{longtable}\par}

% Macro: section (new section for Education, Research Experience, etc.).
\renewcommand{\section}[3]{\\[-1em]\pdfbookmark[2]{#2}{#3}\\%
{\UseSectionFont\raggedright\MakeUppercase{#1}}%
&}

% Macro: subsection.
\renewcommand{\subsection}[3]{\par~\vskip-\baselineskip%
\pdfbookmark[3]{#2}{#3}\par%
{\UseSubSectionFont\raggedright\MakeUppercase{#1}}%
\vspace{0.225\baselineskip}}

% Macro: BigGap, BigGapNoBreak (big vertical gap between items in the same section).
\newcommand{\BigGap}{\\[-1.75mm]~&}
\newcommand{\BigGapNoBreak}{\par\vspace{2.45mm}\par}

% Macro: Gap, GapNoBreak (vertical gap between items in the same section).
\newcommand{\Gap}{\\[-3.5mm]~&}
\newcommand{\GapNoBreak}{\par\vspace{0.7mm}\par}

% Macro: detail (text in smaller font under an item).
\newenvironment{detail}
{\par\begingroup\UseDetailFont}
{\par\endgroup\par}

% Macro: BulletItem.
\newsavebox{\BulletItemIndentation}
\newlength{\BulletItemIndentationWidth}
\newcommand{\BulletItem}{\par%
\savebox{\BulletItemIndentation}{\hspace{1.5mm}\BulletSymbol\hspace{1.25mm}}%
\settowidth{\BulletItemIndentationWidth}{\usebox{\BulletItemIndentation}}%
\noindent\hangafter=1\hangindent=\BulletItemIndentationWidth\ignorespaces%
\usebox{\BulletItemIndentation}\ignorespaces}

% Macro: SubBulletItem.
\newsavebox{\SubBulletItemIndentation}
\newlength{\SubBulletItemIndentationWidth}
\newcommand{\SubBulletItem}{\par%
\savebox{\SubBulletItemIndentation}{\hspace{5.6mm}\SubBulletSymbol\hspace{1.25mm}}%
\settowidth{\SubBulletItemIndentationWidth}{\usebox{\SubBulletItemIndentation}}%
\noindent\hangafter=1\hangindent=\SubBulletItemIndentationWidth\ignorespaces%
\usebox{\SubBulletItemIndentation}\ignorespaces}

% Macro: Item.
\newsavebox{\ItemIndentation}
\newlength{\ItemIndentationWidth}
\newcommand{\Item}{\par%
\savebox{\ItemIndentation}{\hphantom{\hspace{1.5mm}\BulletSymbol\hspace{1.25mm}}}%
\settowidth{\ItemIndentationWidth}{\usebox{\ItemIndentation}}%
\noindent\hangafter=1\hangindent=\ItemIndentationWidth\ignorespaces%
\usebox{\ItemIndentation}\ignorespaces}

% Macro: SubItem.
\newsavebox{\SubItemIndentation}
\newlength{\SubItemIndentationWidth}
\newcommand{\SubItem}{\par%
\savebox{\SubItemIndentation}{\hphantom{\hspace{1.5mm}\BulletSymbol\hspace{1.25mm}}}%
\settowidth{\SubItemIndentationWidth}{\usebox{\SubItemIndentation}}%
\noindent\hangafter=1\hangindent=\SubItemIndentationWidth\ignorespaces%
\usebox{\SubItemIndentation}\ignorespaces}

% Macro: NumberedItem.
\newsavebox{\NumberedItemIndentation}
\newlength{\NumberedItemIndentationWidth}
\newcommand{\NumberedItem}[1]{\par%
\savebox{\NumberedItemIndentation}{{#1}\hspace{2.3mm}}%
\settowidth{\NumberedItemIndentationWidth}{\usebox{\NumberedItemIndentation}}%
\noindent\hangafter=1\hangindent=\NumberedItemIndentationWidth\ignorespaces%
\usebox{\NumberedItemIndentation}\ignorespaces}

% Macro: CharSpace (for aligning single-digit numbers).
\newlength{\CharWidth}
\newcommand{\CharSpace}{\settowidth{\CharWidth}{8}\hspace{\CharWidth}}

% Macro: hide.
\newcommand{\hide}[1]{}


% CV Info (to be customized).
\newcommand{\CVAuthor}{Prakhar Agarwal}
\newcommand{\CVNote}{CV compiled on {\today}}
\newcommand{\CVWebpage}{https://github.com/pkhrag}

% PDF settings and properties.
\hypersetup{
% pdftitle={\CVTitle},
pdfauthor={\CVAuthor},
pdfcreator={XeLaTeX},
pdfproducer={},
pdfkeywords={},
pdfpagemode={},
bookmarks=true,
unicode=true,
bookmarksopen=true,
pdfstartview=FitH,
pdfpagelayout=OneColumn,
pdfpagemode=UseOutlines,
hidelinks,
breaklinks}

% Shorthand.
\newcommand{\CodeCommand}[1]{\mbox{\textbf{\textbackslash{#1}}}}

%%%%%%%%%%%%%%%%%%%%%%%%%%%%%%%%%%%%%%%%%%%%%%%%%%%%%%%%%%%%%%%%%
%% ACTUAL DOCUMENT.
%%%%%%%%%%%%%%%%%%%%%%%%%%%%%%%%%%%%%%%%%%%%%%%%%%%%%%%%%%%%%%%%%

\begin{document}

%%%%%%%%%%%%%%%
% TITLE BLOCK %
%%%%%%%%%%%%%%%

\title{\CVAuthor}

\begin{subtitle}
{SECOND YEAR UNDERGRADUATE · COMPUTER SCIENCE AND ENGINEERING}
\\{INDIAN INSTITUTE OF TECHNOLOGY KANPUR}
\par
\href{mailto:pkhrag@iitk.ac.in}
{pkhrag@iitk.ac.in}
\,\SubBulletSymbol\,
\href{mailto:prakhar.agrwl98@gmail.com}
{prakhar.agrwl98@gmail.com}
\,\SubBulletSymbol\,
(+91)\,9468652335
\,\SubBulletSymbol\,
\href{\CVWebpage}
{\CVWebpage}
\end{subtitle}

\begin{body}

%%%%%%%%%%%%%%%
%% EDUCATION %%
%%%%%%%%%%%%%%%

\section
{Education}
{Education}
{PDF:Education}

{\textbf{Indian Institute of Technology Kanpur}},
UP, India

\GapNoBreak
\BulletItem
Bachelor of Technology, Computer Science and Engineering
\hfill
\DatestampY{2015} --
\DatestampY{2019} [Expected]
\begin{detail}
\SubBulletItem
Cumulative Grade Point/CGPA : 9.5 / 10.0
\end{detail}

\BigGap
{\textbf{Central Public School}},
Rajasthan, India

\GapNoBreak
\BulletItem
12th Grade
\hfill
\DatestampY{2015}
\begin{detail}
\SubBulletItem
RBSE: 89.8\%
\end{detail}

\BigGap
{\textbf{Kendriya Vidyalaya Jhalawar}},
Rajasthan, India

\GapNoBreak
\BulletItem
10th Grade
\hfill
\DatestampY{2013}
\begin{detail}
\SubBulletItem
CBSE | CGPA: 10.0/10.0
\end{detail}

%%%%%%%%%%%%%%%%%%%%%%%%%
%% RESEARCH EXPERIENCE %%
%%%%%%%%%%%%%%%%%%%%%%%%%

% \section
% {Research Experience}
% {Research Experience}
% {PDF:ResearchExperience}

% \href{http://www.example.com/my-institute}
% {\textbf{Institute for Advanced Research}},
% Science College

% \GapNoBreak
% \BulletItem
% Undergraduate Research Student, Science Department
% \hfill
% \DatestampYMD{2004}{05}{15} --
% \DatestampYMD{2005}{05}{15}
% \begin{detail}
% \SubBulletItem
% Project:
% Investigations on the Use of Lasers to Measure Climate Change
% \SubBulletItem
% Supervisors:
% Professor Jane Citizen and
% Dr Ann Yone
% \SubBulletItem
% Research areas:
% Climate change, lasers, statistical analysis, data analytics.
% \end{detail}

%%%%%%%%%%%%%%%%%%
%% PUBLICATIONS %%
%%%%%%%%%%%%%%%%%%

% \section
% {Publications}
% {Publications}
% {PDF:Publications}

% \subsection
% {Journals}
% {Journals}
% {PDF:Journals}

% \GapNoBreak
% \NumberedItem{[11]}
% \href{http://www.example.com/my-paper-doi-5}
% {\underline{J.~Doe}, J.~Citizen, and A.~Yone,
% ``On lasers and climate change,''
% \textit{Journal of Science},
% vol.~89,
% no.~2,
% pp.~4123--4133,
% \DatestampYM{2008}{02}.}

% % Note the use of {\CharSpace} for aligning shorter numbers.
% \Gap
% \NumberedItem{{\CharSpace}[1]}
% \href{http://www.example.com/my-paper-doi-4}
% {\underline{J.~Doe} and J.~Citizen,
% ``Measuring the extent of climate change,''
% \textit{Global Scientific Journal},
% vol.~12,
% no.~4,
% pp.~330--352,
% \DatestampYM{2006}{12}.}

% \BigGap
% \subsection
% {Conferences}
% {Conferences}
% {PDF:Conferences}

% \GapNoBreak
% \NumberedItem{[11]}
% \href{http://www.example.com/my-paper-doi-3}
% {\underline{J.~Doe}, J.~Citizen, and A.~Yone,
% ``On lasers and climate change,''
% in \textit{Proceedings of the Laser Symposium},
% Las Vegas, Nevada, USA,
% \DatestampYM{2007}{01}.}

% \Gap
% \NumberedItem{[10]}
% \href{http://www.example.com/my-paper-doi-2}
% {A.~Yone and \underline{J.~Doe},
% ``Climate change and general relativity,''
% in \textit{Proceedings of the International Astronomical Conference},
% Sydney, Australia,
% \DatestampYM{2006}{8}.}

% % Note the use of {\CharSpace} for aligning shorter numbers.
% \Gap
% \NumberedItem{{\CharSpace}[1]}
% \href{http://www.example.com/my-paper-doi-1}
% {\underline{J.~Doe} and J.~Citizen,
% ``Measuring the extent of climate change,''
% in \textit{Proceedings of the International Climate Change Conference},
% London, UK,
% \DatestampYM{2005}{11}.}

%%%%%%%%%%%%%%%%%%%%%
%% ACADEMIC AWARDS %%
%%%%%%%%%%%%%%%%%%%%%

\section
{Academic Achievements}
{Academic Achievements}
{PDF:AcademicAwards}

\BulletItem
All India Rank - \textbf{126}, JEE-ADVANCE
\hfill
\DatestampY{2015}
% \begin{detail}
% \SubItem
% For attaining a semester GPA of at least 3.75.
% \end{detail}

\Gap
\BulletItem
\textbf{0.1} Percentile, JEE-MAINS
\hfill
\DatestampY{2015}

\Gap
\BulletItem
KVPY Fellow, All India Rank - \textbf{232}
\hfill
\DatestampY{2013}

\Gap
\BulletItem
\textbf{1} Percentile, Physics (NSEP) Olymiad
\hfill
\DatestampY{2014}

\Gap
\BulletItem
Qualified Chemistry (NSEC) Olympiad
\hfill
\DatestampY{2014}


%%%%%%%%%%%%%%%%%%
%% OTHER AWARDS %%
%%%%%%%%%%%%%%%%%%

% \section
% {Other Awards}
% {Other Awards}
% {PDF:OtherAwards}

% \BulletItem
% Chess Tournament,
% First Prize,
% First American University
% \hfill
% \DatestampYMD{2007}{03}{10}
% \begin{detail}
% \SubItem
% Awarded at the Tenth Annual Chess Tournament held during Alumni Weekend.
% \end{detail}

%%%%%%%%%%%%%%%%%%%%%%%%%%%%%%%%%%%%%%%%%%%%
%% PROFESSIONAL AFFILIATIONS & ACTIVITIES %%
%%%%%%%%%%%%%%%%%%%%%%%%%%%%%%%%%%%%%%%%%%%%

% \section
% {Professional Affiliations\newline
% \& Activities}
% {Professional Affiliations \& Activities}
% {PDF:ProfessionalAffiliationsActivities}

% \href{http://www.example.com/my-society}
% {\textbf{Society of Professional Earth Scientists}},
% New York, USA

% \GapNoBreak
% \BulletItem
% Member
% \hfill
% \DatestampY{2009} --
% Present


%%%%%%%%%%%%%%%%%%%%%%%%%%%
%% OTHER WORK EXPERIENCE %%
%%%%%%%%%%%%%%%%%%%%%%%%%%%

\section
{Projects}
{Other Work Experience}
{PDF:WorkExperience}

{\textbf{Autonomous Underwater Vehicle (AUV)}}
\href{https://github.com/AUV-IITK}
{github.com/AUV-IITK}
\hfill
\DatestampYM{2015}{12} --
Present

\GapNoBreak
\BulletItem
Software Developer,
Prof. K.S. Venkatesh \& Prof. Sachin Shinde
\begin{detail}
\SubBulletItem
Member Software Subsystem of team AUV-IITK.
\SubBulletItem
Were placed 2nd at the national level competition SAVe, organized by NIOT Chennai.
\SubBulletItem
Bot can perform task like path following, object detection, dropping objects and shooting\newline topedoes underwater.
\SubBulletItem
Used Robot Operating System (ROS) for Motion Control, OpenCV for Image Processing\newline and Gazebo for simulation.
\end{detail}


\href{https://github.com/pkhrag/ACA-Project-Game-Theory}
{\textbf{Introduction to Game Theory}}
\hfill
\DatestampYM{2016}{01} --
\DatestampYM{2016}{04}

\GapNoBreak
\BulletItem
ACA Semester Project
\begin{detail}
\SubBulletItem
Developed an AI for Tic-Tac-Toe and Connect-4 games.
\SubBulletItem
Used min-max alorithms with alpha-beta optimization and heuristic functions.
\SubBulletItem
Used ZeroMQ library to connect two AI's to play against each other.
\end{detail}

{\textbf{Ethical Hacking}}
\hfill
\DatestampYM{2016}{05} --
\DatestampYM{2016}{07}

\GapNoBreak
\BulletItem
Summer Project Programming Club, IITK
\begin{detail}
\SubBulletItem
Used x86 Assembly Language.
\SubBulletItem
Performed attacks like Buffer Overflow, Reverse Engineering, SQL-Injection, Man In The Middle.
\SubBulletItem
Participated in various CTFs.
\end{detail}
%%%%%%%%%%%%%%%
%% Courses %%
%%%%%%%%%%%%%%%

\section
{Relevent Courses}
{Relevent Courses}
{PDF:ReleventCourses}

\begin{tabular}{l l l}
\textbf{{Completed}}\\
\fontsize{10.7pt}{10.6pt}\selectfont Introduction to Programming & \fontsize{10.7pt}{10.6pt}\selectfont Partial Differential Equation &\fontsize{10.7pt}{10.6pt}\selectfont Linear Algebra\\ 
\fontsize{10.7pt}{10.6pt}\selectfont Introduction to Electrodynamics &\fontsize{10.7pt}{10.6pt}\selectfont Multivariable Calculus &\fontsize{10.7pt}{10.6pt}\selectfont Abstract Algebra\\
\fontsize{10.7pt}{10.6pt}\selectfont Introduction to Classical Mechanics &\fontsize{10.7pt}{10.6pt}\selectfont Discrete Mathematics &\fontsize{10.7pt}{10.6pt}\selectfont Complex Analysis\\
\fontsize{10.7pt}{10.6pt}\selectfont Introduction to Electronics &\fontsize{10.7pt}{10.6pt}\selectfont Logic in Computer Science\\
\textbf{{Ongoing}}\\
\fontsize{10.7pt}{10.6pt}\selectfont Data Structures and Algorithms &\fontsize{10.7pt}{10.6pt}\selectfont Probability and Statistics &\fontsize{10.7pt}{10.6pt}\selectfont Computer Organization\\
\fontsize{10.7pt}{10.6pt}\selectfont Computer System Security &\fontsize{10.7pt}{10.6pt}\selectfont Computer Laboratory &\fontsize{10.7pt}{10.6pt}\selectfont Introduction to Economics\\ 
\end{tabular}
 
%%%%%%%%%%%%
%% SKILLS %%
%%%%%%%%%%%%

\section
{Skills}
{Skills}
{PDF:Skills}

\subsection{Programming}
hh \ C/C++,
Python,
x86 Assembly,
ROS,
OpenCV.

\subsection{Web}
hh\ \ \ \ \ \ \ \ \ \ \ \ \ \ \ \ \ \ \ \ \ \ \  Javascript,
CSS,
HTML.

\subsection{Utilities}
hh\ \ \ \ \ \ \ \ \ \ \ \ \  Linux Shell Utilities,
Git,
GDB,
{\LaTeX},
Sublime,
Vim.

\subsection{Platforms}
hh\ \ \ \ \ \ \ \ \ Linux(Ubuntu),
Windows,
Arduino,
Android.
%%%%%%%%%%%%%%%%%%%%%%%
%% CAMPUS ACTIVITIES %%
%%%%%%%%%%%%%%%%%%%%%%%

\section
{Campus Activities}
{Campus Activities}
{PDF:CampusActivities}

{\textbf{Academic Mentor}}
\hfill
\DatestampYM{2016}{08} --
Present

\GapNoBreak
\BulletItem
Fundamentals of Computing, Counselling Service IITK
\begin{detail}
\SubBulletItem
Helped students to cope up with Academics.
\SubBulletItem
Took Institute level doubt sessions for ESC101 course.
\end{detail}

{\textbf{Departmental Student Body Member}}
\hfill
\DatestampYM{2016}{08} --
Present

\GapNoBreak
\BulletItem
Computer Science and Engineering, IITK
\begin{detail}
\SubBulletItem
Arranged Happy hours and departmental trips.
\end{detail}

{\textbf{Project Mentor}}
\hfill
\DatestampYM{2017}{01} --
Present

\GapNoBreak
\BulletItem
Association for Computing Activities (ACA), IITK
\begin{detail}
\SubBulletItem
Mentoring the project Introduction to Game theory.
\end{detail}

{\textbf{Company Coordinator}}
\hfill
\DatestampYM{2016}{07} --
\DatestampYM{2016}{12}

\GapNoBreak
\BulletItem
Student Placement Office, IITK
\begin{detail}
\SubBulletItem
Conducted various placement activities.
\end{detail}

%%%%%%%%%%%%%%%
%% INTERESTS %%
%%%%%%%%%%%%%%%

\section
{Others}
{Interests}
{PDF:Interests}

\textbf{Presentation Project}
 Diophantine Equations (Under Prof. Nitin Saxena)\newline
\textbf{Google Devfest}
 Time Table Scheduler App\newline
\textbf{Robotrix}
 Made a bot that can pass wedges, lift and stack cubical objects.\newline
\textbf{Mathemania}
 \nth{3} position on Institute level.\newline
\textbf{Project Euler}
 Solved 50+ problems.
%% REFERENCES %%
%%%%%%%%%%%%%%%%b

% \section
% {References}
% {References}
% {PDF:References}

% \BulletItem
% \textbf{Professor Jonathan Public}
% \newline
% Professor of Geology and Mechanical Engineering
% \newline
% First American University
% \begin{detail}
% \SubItem
% 1000 First Avenue, Springfield, Massachusetts 22222, USA
% \newline
% \href{mailto:jonathanpublic@example.com}
% {jonathanpublic@example.com}
% \,\SubBulletSymbol\,
% +1\,(555)\,222-2222
% \end{detail}

% \Gap
% \BulletItem
% \textbf{Dr Alice Bob Carol}
% \newline
% Director, Research \& Development
% \newline
% Alpha Engineering Firm
% \begin{detail}
% \SubItem
% 20 North Street, Oakland, Ohio 33333, USA
% \newline
% \href{mailto:alicebobcarol@example.com}
% {alicebobcarol@example.com}
% \,\SubBulletSymbol\,
% +1\,(555)\,333-3333
% \end{detail}

% %%%%%%%%%%%%%%%%%%%%%%%%%%%%%%%%%
% %% SECTION WITH USAGE EXAMPLES %%
% %%%%%%%%%%%%%%%%%%%%%%%%%%%%%%%%%

% \section
% {Section\newline
% With\newline
% Usage\newline
% Examples}
% {Section With Usage Examples (For PDF Bookmark)}
% {PDF:SectionWithUsageExamples:ForPDFLink}

% \begin{tabular}{l l l}
% \textbf{{Completed}}\\
%  Introduction to Programming & Partial Differential Equation & Linear Algebra\\ 
%  Introduction to Electrodynamics & Multivariable Calculus & Abstract Algebra\\
%  Introduction to Classical Mechanics & Discrete Mathematics & Complex Analysis\\
%  Introduction to Electronics & Logic in Computer Science\\
% \textbf{{Ongoing}}\\
%  Data Structures and Algorithms & Probability and Statistics & Computer Organization\\
%  Computer System Security & Computer Laboratory\\ 
%  \end{tabular}
% \GapNoBreak
% \BulletItem
% Use \CodeCommand{section} and \CodeCommand{subsection} to create sections and subsections.
% These will appear in the PDF bookmarks too.

% \GapNoBreak
% \BulletItem
% This is the second \CodeCommand{BulletItem}.
% Long items are automatically indented.
% Lorem ipsum dolor sit amet, consectetur adipiscing elit.
% Sed sed aliquam massa.
% \begin{detail}
% \SubBulletItem
% This is a \CodeCommand{SubBulletItem}.
% Long items are automatically indented.
% Lorem ipsum dolor sit amet, consectetur adipiscing elit.
% Sed sed aliquam massa.
% Aliquam dignissim mi non enim feugiat elementum.
% Donec sit amet turpis ac velit ultrices volutpat.
% Aliquam vitae elit massa.
% \SubBulletItem
% This is the second \CodeCommand{SubBulletItem}.
% \SubBulletItem
% The \CodeCommand{SubBulletItem}'s are between
% \CodeCommand{begin\{detail\}} and
% \CodeCommand{end\{detail\}} so that they are typeset in a smaller font.
% \end{detail}

% \Gap
% \BulletItem
% This is the third \CodeCommand{BulletItem}.

% \Gap
% \BulletItem
% A \CodeCommand{Gap} or \CodeCommand{GapNoBreak} is inserted between the \CodeCommand{BulletItem}'s so that there is a small vertical space between them.
% The ``NoBreak'' version prevents page breaking, and should be used to avoid orphaned headings and other formatting issues.

% \BigGap
% \subsection
% {This is the Second Subsection}
% {This is the Second Subsection}
% {PDF:ThisIsTheSecondSubSection}

% \GapNoBreak
% \BulletItem
% A \CodeCommand{BigGap} or \CodeCommand{BigGapNoBreak} is inserted between subsections so that there is a bigger vertical space between them.
% The ``NoBreak'' version prevents page breaking.

% %%%%%%%%%%%%%%%%%%%%%%%%%%%%%%%%%%%%%%%%%
% %% ANOTHER SECTION WITH USAGE EXAMPLES %%
% %%%%%%%%%%%%%%%%%%%%%%%%%%%%%%%%%%%%%%%%%

% \section
% {Another\newline
% Section\newline
% With\newline
% Usage\newline
% Examples}
% {Another Section With Usage Examples (For PDF Bookmark)}
% {PDF:AnotherSectionWithUsageExamples:ForPDFLink}

% \textbf{This is a Plain Heading},
% followed by an \CodeCommand{hfill} and a date range
% \hfill
% \DatestampYM{2015}{10} --
% \DatestampYM{2015}{12}

% \GapNoBreak
% \BulletItem
% This is a \CodeCommand{BulletItem}.
% \begin{detail}
% \SubBulletItem
% This is a \CodeCommand{SubBulletItem}.
% \end{detail}

% \GapNoBreak
% \BulletItem
% This is a \CodeCommand{BulletItem}.
% \begin{detail}
% \SubItem
% This is a \CodeCommand{SubItem}, which has no bullet.
% Note the alignment with the \CodeCommand{BulletItem} above.
% \end{detail}

% \GapNoBreak
% \Item
% This is an \CodeCommand{Item}, which has no bullet.
% Note the alignment with the \CodeCommand{BulletItem} above.
% \begin{detail}
% \SubItem
% This is a \CodeCommand{SubItem}.
% \end{detail}

% \GapNoBreak
% \NumberedItem{[16]}
% This is a \CodeCommand{NumberedItem}.
% Note the alignment with the \CodeCommand{SubBulletItem} above.

% \GapNoBreak
% \NumberedItem{{\CharSpace}[6]}
% This is a \CodeCommand{NumberedItem} with a \CodeCommand{CharSpace} in its argument for padding shorter numbers.
% Note the alignment with the \CodeCommand{NumberedItem} above.

% \BigGap
% \textbf{Usage Notes}

% \GapNoBreak
% \BulletItem
% New Lines and Paragraphs
% \begin{detail}
% \SubBulletItem
% To create a new line within the same paragraph (i.e., with the same indentation), use \CodeCommand{newline} instead of \CodeCommand{\textbackslash}.
% The latter will not work because it breaks the long table.
% \SubBulletItem
% To create a new paragraph, use \CodeCommand{par} or simply leave an empty line.
% Paragraph indentations (from
% \CodeCommand{Item},
% \CodeCommand{SubItem},
% \CodeCommand{BulletItem},
% \CodeCommand{SubBulletItem},
% etc.) do not carry across different paragraphs.
% \end{detail}

% \Gap
% \BulletItem
% Vertical Spacing Between Items
% \begin{detail}
% \SubBulletItem
% Use \CodeCommand{Gap} or \CodeCommand{GapNoBreak} to insert a small vertical space between items within the same section.
% \SubBulletItem
% Use \CodeCommand{BigGap} or \CodeCommand{BigGapNoBreak} to insert a bigger vertical space between items within the same section.
% \SubBulletItem
% The ``NoBreak'' versions prevent page breaking.
% \end{detail}

% \Gap
% \BulletItem
% Dates
% \begin{detail}
% \SubBulletItem
% Use
% \CodeCommand{DatestampYMD\{YYYY\}\{MM\}\{DD\}},
% \CodeCommand{DatestampYM\{YYYY\}\{MM\}}, and
% \CodeCommand{DatestampY\{YYYY\}}
% to specify dates.
% \SubBulletItem
% Change the definition of \CodeCommand{DatestampFormatSelection} to adjust how dates are displayed throughout the document (e.g., ``\mbox{2010-12-31}'', ``\mbox{Dec 2010}'', ``\mbox{December 2010}'', ``\mbox{2010}'').
% \end{detail}

\end{body}

%%%%%%%%%%%
% CV NOTE %
%%%%%%%%%%%

\begin{flushright}
\UseNoteFont%
% [\textit{\CVNote}]%
\hspace{2.0mm}\null
\end{flushright}

\label{LastPage}~
\end{document}
